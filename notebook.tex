
% Default to the notebook output style

    


% Inherit from the specified cell style.




    
\documentclass[11pt]{article}

    
    
    \usepackage[T1]{fontenc}
    % Nicer default font (+ math font) than Computer Modern for most use cases
    \usepackage{mathpazo}

    % Basic figure setup, for now with no caption control since it's done
    % automatically by Pandoc (which extracts ![](path) syntax from Markdown).
    \usepackage{graphicx}
    % We will generate all images so they have a width \maxwidth. This means
    % that they will get their normal width if they fit onto the page, but
    % are scaled down if they would overflow the margins.
    \makeatletter
    \def\maxwidth{\ifdim\Gin@nat@width>\linewidth\linewidth
    \else\Gin@nat@width\fi}
    \makeatother
    \let\Oldincludegraphics\includegraphics
    % Set max figure width to be 80% of text width, for now hardcoded.
    \renewcommand{\includegraphics}[1]{\Oldincludegraphics[width=.8\maxwidth]{#1}}
    % Ensure that by default, figures have no caption (until we provide a
    % proper Figure object with a Caption API and a way to capture that
    % in the conversion process - todo).
    \usepackage{caption}
    \DeclareCaptionLabelFormat{nolabel}{}
    \captionsetup{labelformat=nolabel}

    \usepackage{adjustbox} % Used to constrain images to a maximum size 
    \usepackage{xcolor} % Allow colors to be defined
    \usepackage{enumerate} % Needed for markdown enumerations to work
    \usepackage{geometry} % Used to adjust the document margins
    \usepackage{amsmath} % Equations
    \usepackage{amssymb} % Equations
    \usepackage{textcomp} % defines textquotesingle
    % Hack from http://tex.stackexchange.com/a/47451/13684:
    \AtBeginDocument{%
        \def\PYZsq{\textquotesingle}% Upright quotes in Pygmentized code
    }
    \usepackage{upquote} % Upright quotes for verbatim code
    \usepackage{eurosym} % defines \euro
    \usepackage[mathletters]{ucs} % Extended unicode (utf-8) support
    \usepackage[utf8x]{inputenc} % Allow utf-8 characters in the tex document
    \usepackage{fancyvrb} % verbatim replacement that allows latex
    \usepackage{grffile} % extends the file name processing of package graphics 
                         % to support a larger range 
    % The hyperref package gives us a pdf with properly built
    % internal navigation ('pdf bookmarks' for the table of contents,
    % internal cross-reference links, web links for URLs, etc.)
    \usepackage{hyperref}
    \usepackage{longtable} % longtable support required by pandoc >1.10
    \usepackage{booktabs}  % table support for pandoc > 1.12.2
    \usepackage[inline]{enumitem} % IRkernel/repr support (it uses the enumerate* environment)
    \usepackage[normalem]{ulem} % ulem is needed to support strikethroughs (\sout)
                                % normalem makes italics be italics, not underlines
    

    
    
    % Colors for the hyperref package
    \definecolor{urlcolor}{rgb}{0,.145,.698}
    \definecolor{linkcolor}{rgb}{.71,0.21,0.01}
    \definecolor{citecolor}{rgb}{.12,.54,.11}

    % ANSI colors
    \definecolor{ansi-black}{HTML}{3E424D}
    \definecolor{ansi-black-intense}{HTML}{282C36}
    \definecolor{ansi-red}{HTML}{E75C58}
    \definecolor{ansi-red-intense}{HTML}{B22B31}
    \definecolor{ansi-green}{HTML}{00A250}
    \definecolor{ansi-green-intense}{HTML}{007427}
    \definecolor{ansi-yellow}{HTML}{DDB62B}
    \definecolor{ansi-yellow-intense}{HTML}{B27D12}
    \definecolor{ansi-blue}{HTML}{208FFB}
    \definecolor{ansi-blue-intense}{HTML}{0065CA}
    \definecolor{ansi-magenta}{HTML}{D160C4}
    \definecolor{ansi-magenta-intense}{HTML}{A03196}
    \definecolor{ansi-cyan}{HTML}{60C6C8}
    \definecolor{ansi-cyan-intense}{HTML}{258F8F}
    \definecolor{ansi-white}{HTML}{C5C1B4}
    \definecolor{ansi-white-intense}{HTML}{A1A6B2}

    % commands and environments needed by pandoc snippets
    % extracted from the output of `pandoc -s`
    \providecommand{\tightlist}{%
      \setlength{\itemsep}{0pt}\setlength{\parskip}{0pt}}
    \DefineVerbatimEnvironment{Highlighting}{Verbatim}{commandchars=\\\{\}}
    % Add ',fontsize=\small' for more characters per line
    \newenvironment{Shaded}{}{}
    \newcommand{\KeywordTok}[1]{\textcolor[rgb]{0.00,0.44,0.13}{\textbf{{#1}}}}
    \newcommand{\DataTypeTok}[1]{\textcolor[rgb]{0.56,0.13,0.00}{{#1}}}
    \newcommand{\DecValTok}[1]{\textcolor[rgb]{0.25,0.63,0.44}{{#1}}}
    \newcommand{\BaseNTok}[1]{\textcolor[rgb]{0.25,0.63,0.44}{{#1}}}
    \newcommand{\FloatTok}[1]{\textcolor[rgb]{0.25,0.63,0.44}{{#1}}}
    \newcommand{\CharTok}[1]{\textcolor[rgb]{0.25,0.44,0.63}{{#1}}}
    \newcommand{\StringTok}[1]{\textcolor[rgb]{0.25,0.44,0.63}{{#1}}}
    \newcommand{\CommentTok}[1]{\textcolor[rgb]{0.38,0.63,0.69}{\textit{{#1}}}}
    \newcommand{\OtherTok}[1]{\textcolor[rgb]{0.00,0.44,0.13}{{#1}}}
    \newcommand{\AlertTok}[1]{\textcolor[rgb]{1.00,0.00,0.00}{\textbf{{#1}}}}
    \newcommand{\FunctionTok}[1]{\textcolor[rgb]{0.02,0.16,0.49}{{#1}}}
    \newcommand{\RegionMarkerTok}[1]{{#1}}
    \newcommand{\ErrorTok}[1]{\textcolor[rgb]{1.00,0.00,0.00}{\textbf{{#1}}}}
    \newcommand{\NormalTok}[1]{{#1}}
    
    % Additional commands for more recent versions of Pandoc
    \newcommand{\ConstantTok}[1]{\textcolor[rgb]{0.53,0.00,0.00}{{#1}}}
    \newcommand{\SpecialCharTok}[1]{\textcolor[rgb]{0.25,0.44,0.63}{{#1}}}
    \newcommand{\VerbatimStringTok}[1]{\textcolor[rgb]{0.25,0.44,0.63}{{#1}}}
    \newcommand{\SpecialStringTok}[1]{\textcolor[rgb]{0.73,0.40,0.53}{{#1}}}
    \newcommand{\ImportTok}[1]{{#1}}
    \newcommand{\DocumentationTok}[1]{\textcolor[rgb]{0.73,0.13,0.13}{\textit{{#1}}}}
    \newcommand{\AnnotationTok}[1]{\textcolor[rgb]{0.38,0.63,0.69}{\textbf{\textit{{#1}}}}}
    \newcommand{\CommentVarTok}[1]{\textcolor[rgb]{0.38,0.63,0.69}{\textbf{\textit{{#1}}}}}
    \newcommand{\VariableTok}[1]{\textcolor[rgb]{0.10,0.09,0.49}{{#1}}}
    \newcommand{\ControlFlowTok}[1]{\textcolor[rgb]{0.00,0.44,0.13}{\textbf{{#1}}}}
    \newcommand{\OperatorTok}[1]{\textcolor[rgb]{0.40,0.40,0.40}{{#1}}}
    \newcommand{\BuiltInTok}[1]{{#1}}
    \newcommand{\ExtensionTok}[1]{{#1}}
    \newcommand{\PreprocessorTok}[1]{\textcolor[rgb]{0.74,0.48,0.00}{{#1}}}
    \newcommand{\AttributeTok}[1]{\textcolor[rgb]{0.49,0.56,0.16}{{#1}}}
    \newcommand{\InformationTok}[1]{\textcolor[rgb]{0.38,0.63,0.69}{\textbf{\textit{{#1}}}}}
    \newcommand{\WarningTok}[1]{\textcolor[rgb]{0.38,0.63,0.69}{\textbf{\textit{{#1}}}}}
    
    
    % Define a nice break command that doesn't care if a line doesn't already
    % exist.
    \def\br{\hspace*{\fill} \\* }
    % Math Jax compatability definitions
    \def\gt{>}
    \def\lt{<}
    % Document parameters
    \title{Sentiment Score}
    
    
    

    % Pygments definitions
    
\makeatletter
\def\PY@reset{\let\PY@it=\relax \let\PY@bf=\relax%
    \let\PY@ul=\relax \let\PY@tc=\relax%
    \let\PY@bc=\relax \let\PY@ff=\relax}
\def\PY@tok#1{\csname PY@tok@#1\endcsname}
\def\PY@toks#1+{\ifx\relax#1\empty\else%
    \PY@tok{#1}\expandafter\PY@toks\fi}
\def\PY@do#1{\PY@bc{\PY@tc{\PY@ul{%
    \PY@it{\PY@bf{\PY@ff{#1}}}}}}}
\def\PY#1#2{\PY@reset\PY@toks#1+\relax+\PY@do{#2}}

\expandafter\def\csname PY@tok@w\endcsname{\def\PY@tc##1{\textcolor[rgb]{0.73,0.73,0.73}{##1}}}
\expandafter\def\csname PY@tok@c\endcsname{\let\PY@it=\textit\def\PY@tc##1{\textcolor[rgb]{0.25,0.50,0.50}{##1}}}
\expandafter\def\csname PY@tok@cp\endcsname{\def\PY@tc##1{\textcolor[rgb]{0.74,0.48,0.00}{##1}}}
\expandafter\def\csname PY@tok@k\endcsname{\let\PY@bf=\textbf\def\PY@tc##1{\textcolor[rgb]{0.00,0.50,0.00}{##1}}}
\expandafter\def\csname PY@tok@kp\endcsname{\def\PY@tc##1{\textcolor[rgb]{0.00,0.50,0.00}{##1}}}
\expandafter\def\csname PY@tok@kt\endcsname{\def\PY@tc##1{\textcolor[rgb]{0.69,0.00,0.25}{##1}}}
\expandafter\def\csname PY@tok@o\endcsname{\def\PY@tc##1{\textcolor[rgb]{0.40,0.40,0.40}{##1}}}
\expandafter\def\csname PY@tok@ow\endcsname{\let\PY@bf=\textbf\def\PY@tc##1{\textcolor[rgb]{0.67,0.13,1.00}{##1}}}
\expandafter\def\csname PY@tok@nb\endcsname{\def\PY@tc##1{\textcolor[rgb]{0.00,0.50,0.00}{##1}}}
\expandafter\def\csname PY@tok@nf\endcsname{\def\PY@tc##1{\textcolor[rgb]{0.00,0.00,1.00}{##1}}}
\expandafter\def\csname PY@tok@nc\endcsname{\let\PY@bf=\textbf\def\PY@tc##1{\textcolor[rgb]{0.00,0.00,1.00}{##1}}}
\expandafter\def\csname PY@tok@nn\endcsname{\let\PY@bf=\textbf\def\PY@tc##1{\textcolor[rgb]{0.00,0.00,1.00}{##1}}}
\expandafter\def\csname PY@tok@ne\endcsname{\let\PY@bf=\textbf\def\PY@tc##1{\textcolor[rgb]{0.82,0.25,0.23}{##1}}}
\expandafter\def\csname PY@tok@nv\endcsname{\def\PY@tc##1{\textcolor[rgb]{0.10,0.09,0.49}{##1}}}
\expandafter\def\csname PY@tok@no\endcsname{\def\PY@tc##1{\textcolor[rgb]{0.53,0.00,0.00}{##1}}}
\expandafter\def\csname PY@tok@nl\endcsname{\def\PY@tc##1{\textcolor[rgb]{0.63,0.63,0.00}{##1}}}
\expandafter\def\csname PY@tok@ni\endcsname{\let\PY@bf=\textbf\def\PY@tc##1{\textcolor[rgb]{0.60,0.60,0.60}{##1}}}
\expandafter\def\csname PY@tok@na\endcsname{\def\PY@tc##1{\textcolor[rgb]{0.49,0.56,0.16}{##1}}}
\expandafter\def\csname PY@tok@nt\endcsname{\let\PY@bf=\textbf\def\PY@tc##1{\textcolor[rgb]{0.00,0.50,0.00}{##1}}}
\expandafter\def\csname PY@tok@nd\endcsname{\def\PY@tc##1{\textcolor[rgb]{0.67,0.13,1.00}{##1}}}
\expandafter\def\csname PY@tok@s\endcsname{\def\PY@tc##1{\textcolor[rgb]{0.73,0.13,0.13}{##1}}}
\expandafter\def\csname PY@tok@sd\endcsname{\let\PY@it=\textit\def\PY@tc##1{\textcolor[rgb]{0.73,0.13,0.13}{##1}}}
\expandafter\def\csname PY@tok@si\endcsname{\let\PY@bf=\textbf\def\PY@tc##1{\textcolor[rgb]{0.73,0.40,0.53}{##1}}}
\expandafter\def\csname PY@tok@se\endcsname{\let\PY@bf=\textbf\def\PY@tc##1{\textcolor[rgb]{0.73,0.40,0.13}{##1}}}
\expandafter\def\csname PY@tok@sr\endcsname{\def\PY@tc##1{\textcolor[rgb]{0.73,0.40,0.53}{##1}}}
\expandafter\def\csname PY@tok@ss\endcsname{\def\PY@tc##1{\textcolor[rgb]{0.10,0.09,0.49}{##1}}}
\expandafter\def\csname PY@tok@sx\endcsname{\def\PY@tc##1{\textcolor[rgb]{0.00,0.50,0.00}{##1}}}
\expandafter\def\csname PY@tok@m\endcsname{\def\PY@tc##1{\textcolor[rgb]{0.40,0.40,0.40}{##1}}}
\expandafter\def\csname PY@tok@gh\endcsname{\let\PY@bf=\textbf\def\PY@tc##1{\textcolor[rgb]{0.00,0.00,0.50}{##1}}}
\expandafter\def\csname PY@tok@gu\endcsname{\let\PY@bf=\textbf\def\PY@tc##1{\textcolor[rgb]{0.50,0.00,0.50}{##1}}}
\expandafter\def\csname PY@tok@gd\endcsname{\def\PY@tc##1{\textcolor[rgb]{0.63,0.00,0.00}{##1}}}
\expandafter\def\csname PY@tok@gi\endcsname{\def\PY@tc##1{\textcolor[rgb]{0.00,0.63,0.00}{##1}}}
\expandafter\def\csname PY@tok@gr\endcsname{\def\PY@tc##1{\textcolor[rgb]{1.00,0.00,0.00}{##1}}}
\expandafter\def\csname PY@tok@ge\endcsname{\let\PY@it=\textit}
\expandafter\def\csname PY@tok@gs\endcsname{\let\PY@bf=\textbf}
\expandafter\def\csname PY@tok@gp\endcsname{\let\PY@bf=\textbf\def\PY@tc##1{\textcolor[rgb]{0.00,0.00,0.50}{##1}}}
\expandafter\def\csname PY@tok@go\endcsname{\def\PY@tc##1{\textcolor[rgb]{0.53,0.53,0.53}{##1}}}
\expandafter\def\csname PY@tok@gt\endcsname{\def\PY@tc##1{\textcolor[rgb]{0.00,0.27,0.87}{##1}}}
\expandafter\def\csname PY@tok@err\endcsname{\def\PY@bc##1{\setlength{\fboxsep}{0pt}\fcolorbox[rgb]{1.00,0.00,0.00}{1,1,1}{\strut ##1}}}
\expandafter\def\csname PY@tok@kc\endcsname{\let\PY@bf=\textbf\def\PY@tc##1{\textcolor[rgb]{0.00,0.50,0.00}{##1}}}
\expandafter\def\csname PY@tok@kd\endcsname{\let\PY@bf=\textbf\def\PY@tc##1{\textcolor[rgb]{0.00,0.50,0.00}{##1}}}
\expandafter\def\csname PY@tok@kn\endcsname{\let\PY@bf=\textbf\def\PY@tc##1{\textcolor[rgb]{0.00,0.50,0.00}{##1}}}
\expandafter\def\csname PY@tok@kr\endcsname{\let\PY@bf=\textbf\def\PY@tc##1{\textcolor[rgb]{0.00,0.50,0.00}{##1}}}
\expandafter\def\csname PY@tok@bp\endcsname{\def\PY@tc##1{\textcolor[rgb]{0.00,0.50,0.00}{##1}}}
\expandafter\def\csname PY@tok@fm\endcsname{\def\PY@tc##1{\textcolor[rgb]{0.00,0.00,1.00}{##1}}}
\expandafter\def\csname PY@tok@vc\endcsname{\def\PY@tc##1{\textcolor[rgb]{0.10,0.09,0.49}{##1}}}
\expandafter\def\csname PY@tok@vg\endcsname{\def\PY@tc##1{\textcolor[rgb]{0.10,0.09,0.49}{##1}}}
\expandafter\def\csname PY@tok@vi\endcsname{\def\PY@tc##1{\textcolor[rgb]{0.10,0.09,0.49}{##1}}}
\expandafter\def\csname PY@tok@vm\endcsname{\def\PY@tc##1{\textcolor[rgb]{0.10,0.09,0.49}{##1}}}
\expandafter\def\csname PY@tok@sa\endcsname{\def\PY@tc##1{\textcolor[rgb]{0.73,0.13,0.13}{##1}}}
\expandafter\def\csname PY@tok@sb\endcsname{\def\PY@tc##1{\textcolor[rgb]{0.73,0.13,0.13}{##1}}}
\expandafter\def\csname PY@tok@sc\endcsname{\def\PY@tc##1{\textcolor[rgb]{0.73,0.13,0.13}{##1}}}
\expandafter\def\csname PY@tok@dl\endcsname{\def\PY@tc##1{\textcolor[rgb]{0.73,0.13,0.13}{##1}}}
\expandafter\def\csname PY@tok@s2\endcsname{\def\PY@tc##1{\textcolor[rgb]{0.73,0.13,0.13}{##1}}}
\expandafter\def\csname PY@tok@sh\endcsname{\def\PY@tc##1{\textcolor[rgb]{0.73,0.13,0.13}{##1}}}
\expandafter\def\csname PY@tok@s1\endcsname{\def\PY@tc##1{\textcolor[rgb]{0.73,0.13,0.13}{##1}}}
\expandafter\def\csname PY@tok@mb\endcsname{\def\PY@tc##1{\textcolor[rgb]{0.40,0.40,0.40}{##1}}}
\expandafter\def\csname PY@tok@mf\endcsname{\def\PY@tc##1{\textcolor[rgb]{0.40,0.40,0.40}{##1}}}
\expandafter\def\csname PY@tok@mh\endcsname{\def\PY@tc##1{\textcolor[rgb]{0.40,0.40,0.40}{##1}}}
\expandafter\def\csname PY@tok@mi\endcsname{\def\PY@tc##1{\textcolor[rgb]{0.40,0.40,0.40}{##1}}}
\expandafter\def\csname PY@tok@il\endcsname{\def\PY@tc##1{\textcolor[rgb]{0.40,0.40,0.40}{##1}}}
\expandafter\def\csname PY@tok@mo\endcsname{\def\PY@tc##1{\textcolor[rgb]{0.40,0.40,0.40}{##1}}}
\expandafter\def\csname PY@tok@ch\endcsname{\let\PY@it=\textit\def\PY@tc##1{\textcolor[rgb]{0.25,0.50,0.50}{##1}}}
\expandafter\def\csname PY@tok@cm\endcsname{\let\PY@it=\textit\def\PY@tc##1{\textcolor[rgb]{0.25,0.50,0.50}{##1}}}
\expandafter\def\csname PY@tok@cpf\endcsname{\let\PY@it=\textit\def\PY@tc##1{\textcolor[rgb]{0.25,0.50,0.50}{##1}}}
\expandafter\def\csname PY@tok@c1\endcsname{\let\PY@it=\textit\def\PY@tc##1{\textcolor[rgb]{0.25,0.50,0.50}{##1}}}
\expandafter\def\csname PY@tok@cs\endcsname{\let\PY@it=\textit\def\PY@tc##1{\textcolor[rgb]{0.25,0.50,0.50}{##1}}}

\def\PYZbs{\char`\\}
\def\PYZus{\char`\_}
\def\PYZob{\char`\{}
\def\PYZcb{\char`\}}
\def\PYZca{\char`\^}
\def\PYZam{\char`\&}
\def\PYZlt{\char`\<}
\def\PYZgt{\char`\>}
\def\PYZsh{\char`\#}
\def\PYZpc{\char`\%}
\def\PYZdl{\char`\$}
\def\PYZhy{\char`\-}
\def\PYZsq{\char`\'}
\def\PYZdq{\char`\"}
\def\PYZti{\char`\~}
% for compatibility with earlier versions
\def\PYZat{@}
\def\PYZlb{[}
\def\PYZrb{]}
\makeatother


    % Exact colors from NB
    \definecolor{incolor}{rgb}{0.0, 0.0, 0.5}
    \definecolor{outcolor}{rgb}{0.545, 0.0, 0.0}



    
    % Prevent overflowing lines due to hard-to-break entities
    \sloppy 
    % Setup hyperref package
    \hypersetup{
      breaklinks=true,  % so long urls are correctly broken across lines
      colorlinks=true,
      urlcolor=urlcolor,
      linkcolor=linkcolor,
      citecolor=citecolor,
      }
    % Slightly bigger margins than the latex defaults
    
    \geometry{verbose,tmargin=1in,bmargin=1in,lmargin=1in,rmargin=1in}
    
    

    \begin{document}
    
    
    \maketitle
    
    

    
    \hypertarget{sentiment-score}{%
\subsubsection{Sentiment Score}\label{sentiment-score}}

    \begin{Verbatim}[commandchars=\\\{\}]
{\color{incolor}In [{\color{incolor}1}]:} \PY{k+kn}{import} \PY{n+nn}{numpy} \PY{k}{as} \PY{n+nn}{np}
        \PY{k+kn}{import} \PY{n+nn}{pandas} \PY{k}{as} \PY{n+nn}{pd}
        \PY{k+kn}{import} \PY{n+nn}{matplotlib}\PY{n+nn}{.}\PY{n+nn}{pyplot} \PY{k}{as} \PY{n+nn}{plt}
        \PY{k+kn}{import} \PY{n+nn}{seaborn} \PY{k}{as} \PY{n+nn}{sns}
        \PY{n}{sns}\PY{o}{.}\PY{n}{set}\PY{p}{(}\PY{p}{)}
\end{Verbatim}


    Please find a spreadsheet at the following
\href{https://docs.google.com/spreadsheets/d/14XrKWOEYdHNBWA1A3PqMI3ZhI0RsPbi-tD9595wnETs/edit\#gid=0}{URL}:

The spreadsheet contains the following information: - Number of
positive, negative and neutral messages about a certain topic (company,
currency, commodity) at different days.

    \begin{Verbatim}[commandchars=\\\{\}]
{\color{incolor}In [{\color{incolor}2}]:} \PY{n}{st} \PY{o}{=} \PY{n}{pd}\PY{o}{.}\PY{n}{read\PYZus{}csv}\PY{p}{(}\PY{l+s+s1}{\PYZsq{}}\PY{l+s+s1}{sentiment.csv}\PY{l+s+s1}{\PYZsq{}}\PY{p}{)}
\end{Verbatim}


    \begin{Verbatim}[commandchars=\\\{\}]
{\color{incolor}In [{\color{incolor}3}]:} \PY{n}{st}\PY{o}{.}\PY{n}{head}\PY{p}{(}\PY{p}{)}
\end{Verbatim}


\begin{Verbatim}[commandchars=\\\{\}]
{\color{outcolor}Out[{\color{outcolor}3}]:}                   Company         Day Stock Exchange  Total Messages  \textbackslash{}
        0  Fresenius SE \& Co KGaA  2018-02-04            DAX               3   
        1              Merck KGaA  2018-02-07            DAX               3   
        2      Deutsche Boerse AG  2018-02-04            DAX               4   
        3               Munich Re  2018-02-03            DAX               4   
        4                Bayer AG  2018-02-04            DAX               5   
        
           Negative Messages  Neutral Messages  Positive Messages  
        0                  0                 0                  3  
        1                  0                 2                  1  
        2                  1                 0                  3  
        3                  0                 2                  2  
        4                  0                 2                  3  
\end{Verbatim}
            
    \begin{Verbatim}[commandchars=\\\{\}]
{\color{incolor}In [{\color{incolor}4}]:} \PY{n}{st}\PY{o}{.}\PY{n}{info}\PY{p}{(}\PY{p}{)}
\end{Verbatim}


    \begin{Verbatim}[commandchars=\\\{\}]
<class 'pandas.core.frame.DataFrame'>
RangeIndex: 2055 entries, 0 to 2054
Data columns (total 7 columns):
Company              2055 non-null object
Day                  2055 non-null object
Stock Exchange       2055 non-null object
Total Messages       2055 non-null int64
Negative Messages    2055 non-null int64
Neutral Messages     2055 non-null int64
Positive Messages    2055 non-null int64
dtypes: int64(4), object(3)
memory usage: 112.5+ KB

    \end{Verbatim}

    \begin{Verbatim}[commandchars=\\\{\}]
{\color{incolor}In [{\color{incolor}5}]:} \PY{n}{st}\PY{o}{.}\PY{n}{drop}\PY{p}{(}\PY{p}{[}\PY{l+s+s1}{\PYZsq{}}\PY{l+s+s1}{Stock Exchange}\PY{l+s+s1}{\PYZsq{}}\PY{p}{]}\PY{p}{,} \PY{n}{axis}\PY{o}{=}\PY{l+m+mi}{1}\PY{p}{,} \PY{n}{inplace}\PY{o}{=}\PY{k+kc}{True}\PY{p}{)}
\end{Verbatim}


    \begin{itemize}
\tightlist
\item
  In this context, I will denote the number of messages assigned to the
  \texttt{Negative} and \texttt{Positive} categories as \texttt{N} and
  \texttt{P}, respectively, and the total number of messages in all
  categories as \texttt{T}.
\item
  There is also a \texttt{Neutral} category count \texttt{O}, such that
  \texttt{N\ +\ P\ +\ O\ =\ T}.
\end{itemize}

    There are several methods for computing an index from scored sentiment
components of companies. Each is based on comparing positive and
negative messages, and each has advantages and disadvantages.

    \textbf{Absolute Proportional Difference}

\[Sentiment\;Score = \frac{P-N}{P+N+O}\]

    \begin{itemize}
\tightlist
\item
  This formula measure is based on the difference in counts between
  \texttt{positive} and \texttt{negative} messages counts normalized by
  the total number of messages in the company on any topic. From this
  formula it it clear that each count in \texttt{P} or \texttt{N} has
  the same marginal effect: \texttt{1/T} or \texttt{1/(P\ +\ N\ +\ O)}.
\item
  The \texttt{Sentiment\ Score} is equal to zero when there are axactly
  the same number of \texttt{positve} and \texttt{negative} messages,
  \texttt{-1} when there in only one topic on which the company in
  perfectly \texttt{negative} and \texttt{1} when there is one topic and
  the company is perfectly \texttt{positive}.
\end{itemize}

    \begin{Verbatim}[commandchars=\\\{\}]
{\color{incolor}In [{\color{incolor}6}]:} \PY{n}{st}\PY{p}{[}\PY{l+s+s1}{\PYZsq{}}\PY{l+s+s1}{Absolute}\PY{l+s+s1}{\PYZsq{}}\PY{p}{]} \PY{o}{=} \PY{p}{(}\PY{n}{st}\PY{p}{[}\PY{l+s+s1}{\PYZsq{}}\PY{l+s+s1}{Positive Messages}\PY{l+s+s1}{\PYZsq{}}\PY{p}{]} \PY{o}{\PYZhy{}} \PY{n}{st}\PY{p}{[}\PY{l+s+s1}{\PYZsq{}}\PY{l+s+s1}{Negative Messages}\PY{l+s+s1}{\PYZsq{}}\PY{p}{]}\PY{p}{)}\PY{o}{/}\PY{n}{st}\PY{p}{[}\PY{l+s+s1}{\PYZsq{}}\PY{l+s+s1}{Total Messages}\PY{l+s+s1}{\PYZsq{}}\PY{p}{]}
\end{Verbatim}


    \begin{Verbatim}[commandchars=\\\{\}]
{\color{incolor}In [{\color{incolor}7}]:} \PY{n}{st}\PY{o}{.}\PY{n}{head}\PY{p}{(}\PY{l+m+mi}{10}\PY{p}{)}
\end{Verbatim}


\begin{Verbatim}[commandchars=\\\{\}]
{\color{outcolor}Out[{\color{outcolor}7}]:}                   Company         Day  Total Messages  Negative Messages  \textbackslash{}
        0  Fresenius SE \& Co KGaA  2018-02-04               3                  0   
        1              Merck KGaA  2018-02-07               3                  0   
        2      Deutsche Boerse AG  2018-02-04               4                  1   
        3               Munich Re  2018-02-03               4                  0   
        4                Bayer AG  2018-02-04               5                  0   
        5  Fresenius SE \& Co KGaA  2018-02-06               5                  0   
        6               Munich Re  2018-02-04               5                  0   
        7              Merck KGaA  2018-02-04               6                  3   
        8              Merck KGaA  2018-02-08               6                  0   
        9                Bayer AG  2018-02-03               7                  1   
        
           Neutral Messages  Positive Messages  Absolute  
        0                 0                  3  1.000000  
        1                 2                  1  0.333333  
        2                 0                  3  0.500000  
        3                 2                  2  0.500000  
        4                 2                  3  0.600000  
        5                 4                  1  0.200000  
        6                 4                  1  0.200000  
        7                 2                  1 -0.333333  
        8                 3                  3  0.500000  
        9                 2                  4  0.428571  
\end{Verbatim}
            
    The distribution of \texttt{Absolute\ Sentiment\ Score}

    \begin{Verbatim}[commandchars=\\\{\}]
{\color{incolor}In [{\color{incolor}8}]:} \PY{n}{sns}\PY{o}{.}\PY{n}{distplot}\PY{p}{(}\PY{n}{st}\PY{p}{[}\PY{l+s+s1}{\PYZsq{}}\PY{l+s+s1}{Absolute}\PY{l+s+s1}{\PYZsq{}}\PY{p}{]}\PY{p}{,} \PY{n}{bins}\PY{o}{=}\PY{l+m+mi}{20}\PY{p}{)}\PY{p}{;}
\end{Verbatim}


    \begin{center}
    \adjustimage{max size={0.9\linewidth}{0.9\paperheight}}{output_14_0.png}
    \end{center}
    { \hspace*{\fill} \\}
    
    \begin{itemize}
\tightlist
\item
  The disadvantage of this formula: a sentiment score is affected by
  non-sentiment (\texttt{Neutral} messages). In order to address this
  problem, I propose an alternative measure that \emph{relative
  proportional difference}.
\end{itemize}

    \textbf{Relative Proportional Difference}
\[Sentiment\;Score = \frac{P-N}{P+N}\]

    \begin{itemize}
\tightlist
\item
  This formula also ranges from -1 to 1, but makes explict the range
  constrain hidden of \texttt{Absolute\ Proportional\ Difference} by
  non-sentiment (\texttt{Neutal} messages).
\item
  Dividing by \texttt{P\ +\ N} the formula from variation in the
  sentimented messages assigns to any topic. The only remaining
  influence of sentimented message is that the overall number of
  messages available to show a sentiment is increased or reduced.
\item
  Finally, unlike \texttt{Absolute\ Proportional\ Difference} this
  formula will not nesessarily create an apparent move to a more
  centrist position (zero) if the proportion of \texttt{Neutral}
  messages are dominance.
\end{itemize}

    \begin{Verbatim}[commandchars=\\\{\}]
{\color{incolor}In [{\color{incolor}9}]:} \PY{n}{st}\PY{p}{[}\PY{l+s+s1}{\PYZsq{}}\PY{l+s+s1}{Relative}\PY{l+s+s1}{\PYZsq{}}\PY{p}{]} \PY{o}{=} \PY{p}{(}\PY{n}{st}\PY{p}{[}\PY{l+s+s1}{\PYZsq{}}\PY{l+s+s1}{Positive Messages}\PY{l+s+s1}{\PYZsq{}}\PY{p}{]} \PY{o}{\PYZhy{}} \PY{n}{st}\PY{p}{[}\PY{l+s+s1}{\PYZsq{}}\PY{l+s+s1}{Negative Messages}\PY{l+s+s1}{\PYZsq{}}\PY{p}{]}\PY{p}{)}\PY{o}{/}\PY{p}{(}\PY{n}{st}\PY{p}{[}\PY{l+s+s1}{\PYZsq{}}\PY{l+s+s1}{Positive Messages}\PY{l+s+s1}{\PYZsq{}}\PY{p}{]} \PY{o}{+} \PY{n}{st}\PY{p}{[}\PY{l+s+s1}{\PYZsq{}}\PY{l+s+s1}{Negative Messages}\PY{l+s+s1}{\PYZsq{}}\PY{p}{]}\PY{p}{)}
\end{Verbatim}


    \begin{Verbatim}[commandchars=\\\{\}]
{\color{incolor}In [{\color{incolor}10}]:} \PY{n}{st}\PY{o}{.}\PY{n}{head}\PY{p}{(}\PY{l+m+mi}{10}\PY{p}{)}
\end{Verbatim}


\begin{Verbatim}[commandchars=\\\{\}]
{\color{outcolor}Out[{\color{outcolor}10}]:}                   Company         Day  Total Messages  Negative Messages  \textbackslash{}
         0  Fresenius SE \& Co KGaA  2018-02-04               3                  0   
         1              Merck KGaA  2018-02-07               3                  0   
         2      Deutsche Boerse AG  2018-02-04               4                  1   
         3               Munich Re  2018-02-03               4                  0   
         4                Bayer AG  2018-02-04               5                  0   
         5  Fresenius SE \& Co KGaA  2018-02-06               5                  0   
         6               Munich Re  2018-02-04               5                  0   
         7              Merck KGaA  2018-02-04               6                  3   
         8              Merck KGaA  2018-02-08               6                  0   
         9                Bayer AG  2018-02-03               7                  1   
         
            Neutral Messages  Positive Messages  Absolute  Relative  
         0                 0                  3  1.000000       1.0  
         1                 2                  1  0.333333       1.0  
         2                 0                  3  0.500000       0.5  
         3                 2                  2  0.500000       1.0  
         4                 2                  3  0.600000       1.0  
         5                 4                  1  0.200000       1.0  
         6                 4                  1  0.200000       1.0  
         7                 2                  1 -0.333333      -0.5  
         8                 3                  3  0.500000       1.0  
         9                 2                  4  0.428571       0.6  
\end{Verbatim}
            
    \begin{Verbatim}[commandchars=\\\{\}]
{\color{incolor}In [{\color{incolor}11}]:} \PY{n}{st}\PY{p}{[}\PY{l+s+s1}{\PYZsq{}}\PY{l+s+s1}{Relative}\PY{l+s+s1}{\PYZsq{}}\PY{p}{]}\PY{o}{.}\PY{n}{isna}\PY{p}{(}\PY{p}{)}\PY{o}{.}\PY{n}{sum}\PY{p}{(}\PY{p}{)}
\end{Verbatim}


\begin{Verbatim}[commandchars=\\\{\}]
{\color{outcolor}Out[{\color{outcolor}11}]:} 313
\end{Verbatim}
            
    The distribution of \texttt{Relative\ Proportional\ Difference}

    \begin{Verbatim}[commandchars=\\\{\}]
{\color{incolor}In [{\color{incolor}12}]:} \PY{n}{Relative} \PY{o}{=} \PY{n}{st}\PY{p}{[}\PY{l+s+s1}{\PYZsq{}}\PY{l+s+s1}{Relative}\PY{l+s+s1}{\PYZsq{}}\PY{p}{]}\PY{o}{.}\PY{n}{dropna}\PY{p}{(}\PY{p}{)}
         \PY{n}{sns}\PY{o}{.}\PY{n}{distplot}\PY{p}{(}\PY{n}{Relative}\PY{p}{,} \PY{n}{bins}\PY{o}{=}\PY{l+m+mi}{20}\PY{p}{)}\PY{p}{;}
\end{Verbatim}


    \begin{center}
    \adjustimage{max size={0.9\linewidth}{0.9\paperheight}}{output_22_0.png}
    \end{center}
    { \hspace*{\fill} \\}
    
    \begin{itemize}
\item
  Although \texttt{Relative\ Proportional\ Difference} appears to fix
  the problems of messages in \texttt{Neutral} category effecting
  position estimates, but sill have a disadvantages. A sentiment score
  may tend to cluster very strongly near the scale endpoints (-1 or 1)
  because they may contain content primarily or exclusively of either
  positive or negative.
\item
  This has the unfortunate effect of forcing the
  \texttt{Sentiment\ Score} to -1 when \texttt{P} = 0 irrespective of
  the value of \texttt{N}, or to 1 when \texttt{N} = 0 irrespective of
  the value of \texttt{P}.
\end{itemize}

    \textbf{Log Odds-Ratios (OR)} - Because we are primarily interested in
how the company more positive (or negative) a topic on particular day or
between companies, we view it as most natural to consider proportional
changes on a symmetrical \texttt{positive-negative} scale. The natural
measure for this purpose in the empirical logit:

\[
Sentiment\;Score  = log \frac{P+0.5}{N+0.5} 
= log(P+0.5) - log(N+0.5) 
\]

\begin{itemize}
\tightlist
\item
  Like \texttt{Relative\ Proportional\ Difference}, \texttt{Log\ OR} is
  conditional because it only considers messages that are assigned to
  \texttt{positive} or \texttt{negative}. Unlike
  \texttt{Absolute\ Proportional\ Difference} and
  \texttt{Relative\ Proportional\ Difference}, however, the
  \texttt{Log\ OR} has no predefined endpoints (like {[}-1,1{]}): it is
  possible to generate sentiment of any level of extremity. In this
  respect, \texttt{Log\ OR} better reflects the sentiment of different
  companies, and also the companies that have many messages per day to
  companies that have few messages per day.
\item
  This tends to have the smoothest properties and is symmetric around
  zero. The 0.5 is a smoother to prevent \texttt{log(0)}.
\end{itemize}

    \begin{Verbatim}[commandchars=\\\{\}]
{\color{incolor}In [{\color{incolor}13}]:} \PY{n}{st}\PY{p}{[}\PY{l+s+s1}{\PYZsq{}}\PY{l+s+s1}{Log OR}\PY{l+s+s1}{\PYZsq{}}\PY{p}{]} \PY{o}{=} \PY{n}{np}\PY{o}{.}\PY{n}{log}\PY{p}{(}\PY{n}{st}\PY{p}{[}\PY{l+s+s1}{\PYZsq{}}\PY{l+s+s1}{Positive Messages}\PY{l+s+s1}{\PYZsq{}}\PY{p}{]} \PY{o}{+} \PY{l+m+mf}{0.5}\PY{p}{)} \PY{o}{\PYZhy{}} \PY{n}{np}\PY{o}{.}\PY{n}{log}\PY{p}{(}\PY{n}{st}\PY{p}{[}\PY{l+s+s1}{\PYZsq{}}\PY{l+s+s1}{Negative Messages}\PY{l+s+s1}{\PYZsq{}}\PY{p}{]} \PY{o}{+} \PY{l+m+mf}{0.5}\PY{p}{)}
\end{Verbatim}


    \begin{Verbatim}[commandchars=\\\{\}]
{\color{incolor}In [{\color{incolor}14}]:} \PY{n}{st}\PY{o}{.}\PY{n}{head}\PY{p}{(}\PY{l+m+mi}{10}\PY{p}{)}
\end{Verbatim}


\begin{Verbatim}[commandchars=\\\{\}]
{\color{outcolor}Out[{\color{outcolor}14}]:}                   Company         Day  Total Messages  Negative Messages  \textbackslash{}
         0  Fresenius SE \& Co KGaA  2018-02-04               3                  0   
         1              Merck KGaA  2018-02-07               3                  0   
         2      Deutsche Boerse AG  2018-02-04               4                  1   
         3               Munich Re  2018-02-03               4                  0   
         4                Bayer AG  2018-02-04               5                  0   
         5  Fresenius SE \& Co KGaA  2018-02-06               5                  0   
         6               Munich Re  2018-02-04               5                  0   
         7              Merck KGaA  2018-02-04               6                  3   
         8              Merck KGaA  2018-02-08               6                  0   
         9                Bayer AG  2018-02-03               7                  1   
         
            Neutral Messages  Positive Messages  Absolute  Relative    Log OR  
         0                 0                  3  1.000000       1.0  1.945910  
         1                 2                  1  0.333333       1.0  1.098612  
         2                 0                  3  0.500000       0.5  0.847298  
         3                 2                  2  0.500000       1.0  1.609438  
         4                 2                  3  0.600000       1.0  1.945910  
         5                 4                  1  0.200000       1.0  1.098612  
         6                 4                  1  0.200000       1.0  1.098612  
         7                 2                  1 -0.333333      -0.5 -0.847298  
         8                 3                  3  0.500000       1.0  1.945910  
         9                 2                  4  0.428571       0.6  1.098612  
\end{Verbatim}
            
    Which company has the most negative score?

    \begin{Verbatim}[commandchars=\\\{\}]
{\color{incolor}In [{\color{incolor}15}]:} \PY{n+nb}{min}\PY{p}{(}\PY{n}{st}\PY{p}{[}\PY{l+s+s1}{\PYZsq{}}\PY{l+s+s1}{Log OR}\PY{l+s+s1}{\PYZsq{}}\PY{p}{]}\PY{p}{)}
\end{Verbatim}


\begin{Verbatim}[commandchars=\\\{\}]
{\color{outcolor}Out[{\color{outcolor}15}]:} -4.056988775678332
\end{Verbatim}
            
    \begin{Verbatim}[commandchars=\\\{\}]
{\color{incolor}In [{\color{incolor}16}]:} \PY{n}{st}\PY{o}{.}\PY{n}{loc}\PY{p}{[}\PY{n}{st}\PY{p}{[}\PY{l+s+s1}{\PYZsq{}}\PY{l+s+s1}{Log OR}\PY{l+s+s1}{\PYZsq{}}\PY{p}{]} \PY{o}{==} \PY{o}{\PYZhy{}}\PY{l+m+mf}{4.056988775678332}\PY{p}{]}
\end{Verbatim}


\begin{Verbatim}[commandchars=\\\{\}]
{\color{outcolor}Out[{\color{outcolor}16}]:}     Company         Day  Total Messages  Negative Messages  Neutral Messages  \textbackslash{}
         787   GoPro  2018-02-03             150                144                 4   
         
              Positive Messages  Absolute  Relative    Log OR  
         787                  2 -0.946667 -0.972603 -4.056989  
\end{Verbatim}
            
    Which company has the most positive score?

    \begin{Verbatim}[commandchars=\\\{\}]
{\color{incolor}In [{\color{incolor}17}]:} \PY{n+nb}{max}\PY{p}{(}\PY{n}{st}\PY{p}{[}\PY{l+s+s1}{\PYZsq{}}\PY{l+s+s1}{Log OR}\PY{l+s+s1}{\PYZsq{}}\PY{p}{]}\PY{p}{)}
\end{Verbatim}


\begin{Verbatim}[commandchars=\\\{\}]
{\color{outcolor}Out[{\color{outcolor}17}]:} 6.1675164908883415
\end{Verbatim}
            
    \begin{Verbatim}[commandchars=\\\{\}]
{\color{incolor}In [{\color{incolor}18}]:} \PY{n}{st}\PY{o}{.}\PY{n}{loc}\PY{p}{[}\PY{n}{st}\PY{p}{[}\PY{l+s+s1}{\PYZsq{}}\PY{l+s+s1}{Log OR}\PY{l+s+s1}{\PYZsq{}}\PY{p}{]} \PY{o}{==} \PY{l+m+mf}{6.1675164908883415}\PY{p}{]}
\end{Verbatim}


\begin{Verbatim}[commandchars=\\\{\}]
{\color{outcolor}Out[{\color{outcolor}18}]:}              Company         Day  Total Messages  Negative Messages  \textbackslash{}
         665  First Solar Inc  2018-02-05             312                  0   
         
              Neutral Messages  Positive Messages  Absolute  Relative    Log OR  
         665                74                238  0.762821       1.0  6.167516  
\end{Verbatim}
            
    The distribution of \texttt{Log\ Odds-Ratios\ (OR)}

    \begin{Verbatim}[commandchars=\\\{\}]
{\color{incolor}In [{\color{incolor}19}]:} \PY{n}{sns}\PY{o}{.}\PY{n}{distplot}\PY{p}{(}\PY{n}{st}\PY{p}{[}\PY{l+s+s1}{\PYZsq{}}\PY{l+s+s1}{Log OR}\PY{l+s+s1}{\PYZsq{}}\PY{p}{]}\PY{p}{,} \PY{n}{bins}\PY{o}{=}\PY{l+m+mi}{30}\PY{p}{)}\PY{p}{;}
\end{Verbatim}


    \begin{center}
    \adjustimage{max size={0.9\linewidth}{0.9\paperheight}}{output_34_0.png}
    \end{center}
    { \hspace*{\fill} \\}
    
    My conclusions are that the use of \texttt{Logit\ Odds-Ratios} formula
to estimate positively (or negatively) from counts of messages
categories, as well as to my demonstration through direct comparison to
other formula, suggest that the \texttt{Logit\ Odds-Ratios} is superior
and should be used for calculating a sentiment score.

    \textbf{Top ten companies that have most number of total messages}

    \begin{Verbatim}[commandchars=\\\{\}]
{\color{incolor}In [{\color{incolor}20}]:} \PY{c+c1}{\PYZsh{} st[\PYZsq{}Total Messages\PYZsq{}].groupby(st[\PYZsq{}Company\PYZsq{}]).describe().sort\PYZus{}values(\PYZsq{}mean\PYZsq{}, ascending=False).head(10)}
         \PY{n}{st\PYZus{}gb} \PY{o}{=} \PY{n}{st}\PY{o}{.}\PY{n}{groupby}\PY{p}{(}\PY{l+s+s1}{\PYZsq{}}\PY{l+s+s1}{Company}\PY{l+s+s1}{\PYZsq{}}\PY{p}{)}\PY{p}{[}\PY{p}{[}\PY{l+s+s1}{\PYZsq{}}\PY{l+s+s1}{Total Messages}\PY{l+s+s1}{\PYZsq{}}\PY{p}{,}\PY{l+s+s1}{\PYZsq{}}\PY{l+s+s1}{Negative Messages}\PY{l+s+s1}{\PYZsq{}}\PY{p}{,}\PY{l+s+s1}{\PYZsq{}}\PY{l+s+s1}{Neutral Messages}\PY{l+s+s1}{\PYZsq{}}\PY{p}{,}\PY{l+s+s1}{\PYZsq{}}\PY{l+s+s1}{Positive Messages}\PY{l+s+s1}{\PYZsq{}}\PY{p}{]}\PY{p}{]}\PY{o}{.}\PY{n}{sum}\PY{p}{(}\PY{p}{)}\PY{o}{.}\PY{n}{sort\PYZus{}values}\PY{p}{(}\PY{n}{by}\PY{o}{=}\PY{l+s+s1}{\PYZsq{}}\PY{l+s+s1}{Total Messages}\PY{l+s+s1}{\PYZsq{}}\PY{p}{,} \PY{n}{ascending}\PY{o}{=}\PY{k+kc}{False}\PY{p}{)}\PY{o}{.}\PY{n}{head}\PY{p}{(}\PY{l+m+mi}{10}\PY{p}{)}
\end{Verbatim}


    \begin{Verbatim}[commandchars=\\\{\}]
{\color{incolor}In [{\color{incolor}21}]:} \PY{n}{st\PYZus{}gb}\PY{o}{.}\PY{n}{head}\PY{p}{(}\PY{p}{)}
\end{Verbatim}


\begin{Verbatim}[commandchars=\\\{\}]
{\color{outcolor}Out[{\color{outcolor}21}]:}                   Total Messages  Negative Messages  Neutral Messages  \textbackslash{}
         Company                                                                 
         Apple Inc                  37217               7160             16075   
         Alphabet Inc               28568               9625             11492   
         Deutsche Bank AG           22146               9210              8625   
         Wells Fargo \& Co           21439              11462              4076   
         Twitter Inc                20805               6759              4791   
         
                           Positive Messages  
         Company                              
         Apple Inc                     13982  
         Alphabet Inc                   7451  
         Deutsche Bank AG               4311  
         Wells Fargo \& Co               5901  
         Twitter Inc                    9255  
\end{Verbatim}
            
    \begin{Verbatim}[commandchars=\\\{\}]
{\color{incolor}In [{\color{incolor}22}]:} \PY{n}{dt} \PY{o}{=} \PY{n}{st\PYZus{}gb}\PY{o}{.}\PY{n}{transpose}\PY{p}{(}\PY{p}{)}
\end{Verbatim}


    \begin{Verbatim}[commandchars=\\\{\}]
{\color{incolor}In [{\color{incolor}23}]:} \PY{n}{dt\PYZus{}bar} \PY{o}{=} \PY{n}{dt}\PY{o}{.}\PY{n}{reset\PYZus{}index}\PY{p}{(}\PY{p}{)}\PY{o}{.}\PY{n}{melt}\PY{p}{(}\PY{n}{id\PYZus{}vars}\PY{o}{=}\PY{p}{[}\PY{l+s+s2}{\PYZdq{}}\PY{l+s+s2}{index}\PY{l+s+s2}{\PYZdq{}}\PY{p}{]}\PY{p}{)}
         \PY{n}{dt\PYZus{}bar}\PY{o}{.}\PY{n}{head}\PY{p}{(}\PY{p}{)}
\end{Verbatim}


\begin{Verbatim}[commandchars=\\\{\}]
{\color{outcolor}Out[{\color{outcolor}23}]:}                index       Company  value
         0     Total Messages     Apple Inc  37217
         1  Negative Messages     Apple Inc   7160
         2   Neutral Messages     Apple Inc  16075
         3  Positive Messages     Apple Inc  13982
         4     Total Messages  Alphabet Inc  28568
\end{Verbatim}
            
    \begin{Verbatim}[commandchars=\\\{\}]
{\color{incolor}In [{\color{incolor}24}]:} \PY{n}{plt}\PY{o}{.}\PY{n}{figure}\PY{p}{(}\PY{n}{figsize}\PY{o}{=}\PY{p}{(}\PY{l+m+mi}{20}\PY{p}{,}\PY{l+m+mi}{9}\PY{p}{)}\PY{p}{)}
         \PY{n}{sns}\PY{o}{.}\PY{n}{barplot}\PY{p}{(}\PY{n}{x}\PY{o}{=}\PY{l+s+s2}{\PYZdq{}}\PY{l+s+s2}{Company}\PY{l+s+s2}{\PYZdq{}}\PY{p}{,} \PY{n}{y}\PY{o}{=}\PY{l+s+s2}{\PYZdq{}}\PY{l+s+s2}{value}\PY{l+s+s2}{\PYZdq{}}\PY{p}{,} \PY{n}{hue}\PY{o}{=}\PY{l+s+s2}{\PYZdq{}}\PY{l+s+s2}{index}\PY{l+s+s2}{\PYZdq{}}\PY{p}{,} \PY{n}{data}\PY{o}{=}\PY{n}{dt\PYZus{}bar}\PY{p}{)}\PY{p}{;}
\end{Verbatim}


    \begin{center}
    \adjustimage{max size={0.9\linewidth}{0.9\paperheight}}{output_41_0.png}
    \end{center}
    { \hspace*{\fill} \\}
    

    % Add a bibliography block to the postdoc
    
    
    
    \end{document}
